\documentclass[aps,preprintnumbers,showpacs,nofootinbib,superscriptaddress,floatfix]{revtex4}

\pdfoutput=1 % if your are submitting a pdflatex (i.e. if you have
             % images in pdf, png or jpg format)

%%%%%%%%%%%%%%%%%%%%%%%%%%%%
\usepackage{graphicx}
\usepackage{amssymb}
\usepackage{amsmath}
\usepackage{bm}
\usepackage{slashed}
\usepackage{datetime}
\usepackage{mciteplus}
\usepackage{multirow}
\usepackage{color}
%%%%%%%%%%%%%%%%%%%%%%%%%%%

\newcommand{\xbj}{x_B}
\newcommand{\zh}{z_h}

%% macro to draw particularly short right arrows. Choose one of the two
\newcommand{\smarrow}{\mbox{\raisebox{-4.5pt}[0pt][0pt]{$\hspace{-1pt} 
		\vec{\phantom{v}}$}}}
%\newcommand{\smarrow}{\mbox{\raisebox{0.5pt}{\tiny $\hspace{-1.5pt} \rightarrow 
%\hspace{-8.25pt}{\color{white} \rule[1pt]{1.5pt}{0.5pt}} \hspace{4pt} $}} }

%% names of experiments + MINUIT
\newcommand{\hermes}{\textsc{Hermes }}
\newcommand{\compass}{\textsc{Compass }}
\newcommand{\minuit}{\textsc{Minuit }}

%% macro to manage T versus perp notation
\newcommand{\T}{\perp}
\newcommand{\Tperp}{T}
\newcommand{\bT}{\zeta_T}

%%%%%%%%%%%%%%%%%%%%%%%%%%%%%%%%%%%%%%%%%%%%%%%

\begin{document}
\allowdisplaybreaks[2]


\title{
Extraction of partonic transverse momentum distributions 
from semi-inclusive deep
inelastic scattering and Drell-Yan data
}

\author{Alessandro Bacchetta}
\email{alessandro.bacchetta@unipv.it}
\affiliation{Dipartimento di Fisica, Universit\`a di Pavia, via Bassi 6,
  I-27100 Pavia} 
\affiliation{INFN Sezione di Pavia, via Bassi 6, I-27100 Pavia, Italy}

\author{Filippo Delcarro}
\email{filippo.delcarro01@ateneopv.it}
\affiliation{Dipartimento di Fisica, Universit\`a di Pavia, via Bassi 6,
  I-27100 Pavia} 
\affiliation{INFN Sezione di Pavia, via Bassi 6, I-27100 Pavia, Italy}

\author{Cristian Pisano}
\email{alessandro.bacchetta@unipv.it}
\affiliation{Dipartimento di Fisica, Universit\`a di Pavia, via Bassi 6,
  I-27100 Pavia} 
\affiliation{INFN Sezione di Pavia, via Bassi 6, I-27100 Pavia, Italy}

\author{Marco Radici}
\email{marco.radici@pv.infn.it}
\affiliation{INFN Sezione di Pavia, via Bassi 6, I-27100 Pavia, Italy}

\author{Andrea Signori}
\email{asignori@jlab.org}
\affiliation{Theory Center, Thomas Jefferson National Accelerator Facility, 12000 Jefferson Avenue, Newport News, VA 23606, USA}
\affiliation{Department of Physics and Astronomy, VU University Amsterdam, De
  Boelelaan 1081, NL-1081 HV Amsterdam, the Netherlands}
\affiliation{Nikhef, Science Park 105, NL-1098 XG Amsterdam, the
  Netherlands}

\begin{abstract}
We present an extraction of unpolarized partonic transverse momentum distributions (TMDs)
from a simultaneous fit of available data measured in semi-inclusive deep inelastic scattering 
\textcolor{red}{and in Drell-Yan processes through the production of photon and $Z$ bosons}. 
To connect data at different scales, we use TMD evolution at next-to-leading logarithmic accuracy. The
analysis is restricted to the low-transverse-momentum region, with no matching
to fixed-order calculations at high transverse momentum. We introduce specific
choices to deal with TMD evolution at low scales, of the order of 1 GeV$^2$.
This could be considered as a first attempt at a global fit of TMDs.
\end{abstract}

\preprint{JLAB THY 17-****}

\date{\today, \currenttime}

\pacs{13.60.Le, 13.87.Fh,14.20.Dh}

\maketitle

%%%%%%%%%%%%%%%%%%%%%%%%%%%%%%%%%%%%%%%%%%%%%%%%%%%%%%%%%%%%%%%%%%
\section{Introduction}
\label{s:intro}
%%%%%%%%%%%%%%%%%%%%%%%%%%%%%%%%%%%%%%%%%%%%%%%%%%%%%%%%%%%%%%%%%%

Parton distribution functions describe the internal structure of the nucleon
in terms of its elementary constituents (quarks and gluons). They cannot be
easily computed from first principles, because they require the ability to
carry out Quantum Chromodynamics (QCD) calculations in a nonperturbative
regime. Experimental observables in any hard scattering experiment
involving hadrons are related to parton distribution functions \textcolor{red}{(PDFs) and fragmentation functions (FFs)} , in a way that is specified by factorization theorems \textcolor{red}{(see, e.g., Refs.~\cite{Collins:1989gx,Collins:2011zzd})}. These theorems also elucidate the universality properties of \textcolor{red}{PDFs and FFs} (i.e., the fact that they are the same in different processes) and their evolution equations (i.e., how they get modified by the change in
the hard scale of the process). Availability of measurements \textcolor{red}{of different processes in} different
experiments makes it possible to test the reliability of factorization theorems and extract \textcolor{red}{PDFs and FFs} through so-called global fits. On the other side, the knowledge of \textcolor{red}{PDFs and FFs} allow us
to make predictions for hard hadronic processes. These general statements apply equally well to
standard collinear \textcolor{red}{PDFs and FFs} and to transverse-momentum-dependent parton distribution functions (TMD \textcolor{red}{PDFs) and fragmentation functions (TMD FFs). The} PDFs
describe the distribution of partons \textcolor{red}{when they move collinear with the parent hadron;} \textcolor{blue}{(or: for which the light-cone minus and transverse components of the momentum are averaged)} \textcolor{red}{hence, PDFs are function only of the parton longitudinal momentum fraction $x$. TMD PDFs} include also the dependence on transverse momentum components \textcolor{red}{$k_{\T}^2$}. They can be interpreted as three-dimensional generalizations of standard PDFs. \textcolor{red}{Similar arguments apply to FFs and TMD FFs. }

Apart from the many similarities, there are also several differences between collinear and TMD distributions. From the formal point of view, factorization theorems for the two types of functions are qualitatively different, implying also different universality properties and evolution equations~\cite{Rogers:2015sqa}. From the experimental point
of view, observables related to TMDs require the measurement of some transverse
momentum component much smaller than the hard scale of the process~\cite{Bacchetta:2016ccz,Radici:2016hbh}. For
instance, \textcolor{red}{Deep-Inelastic Scattering (DIS) is characterized by a hard scale represented by the
4-momentum squared of the virtual photon ($-Q^2$). In inclusive DIS this is the only scale of the process, and access is limited to collinear PDFs and FFs. In semi-inclusive DIS (SIDIS) also the transverse momentum of the outgoing detected 
hadron ($P_{hT}$) can be measured~\cite{Mulders:1995dh,Bacchetta:2006tn}. If $P_{hT}^2\ll Q^2$, TMD
factorization can be applied and the process is sensitive to TMDs~\cite{Collins:2011zzd}. }

%If $P_{h\perp}^2\sim Q^2$ or if the measurement is integrated over $P_{h\perp}$,
%collinear factorization applies and the process is sensitive to collinear
%PDFs. 

If polarization is taken into account, several TMDs can be introduced~\cite{Mulders:1995dh,Bacchetta:2000jk,Boer:2016xqr} and possibly extracted from measurements~\cite{Angeles-Martinez:2015sea,Aschenauer:2015ndk,Boglione:2015zyc}. In this work, we focus on the simplest ones, i.e., the unpolarized TMD \textcolor{red}{PDF} $f_1^q(x,k_{\T}^2)$ and the unpolarized TMD \textcolor{red}{FF} \textcolor{red}{$D_1^{q \to h}(z,P_{hT}^2)$, where $z$ is the fractional energy carried by the detected hadron $h$. Despite their simplicity, the phenomenology of these unpolarized TMDs present several challenges~\cite{Signori:2016lvd}: the functional form of TMDs at low partonic transverse momentum, its possible dependence on the parton flavor~\cite{Signori:2013mda}, the implementation of TMD evolution~\cite{Bacchetta:2015ora,Rogers:2015sqa}, the matching to fixed-order calculations in collinear factorization~\cite{Collins:2016hqq}. } 

We take into consideration three kinds of processes: \textcolor{red}{SIDIS, and Drell--Yan processes (DY) with the production of virtual photons and $Z$ bosons.} To date, they represent \textcolor{blue}{almost} all possible processes where experimental information is available for unpolarized TMD extractions. The only important
process currently missing is electron-positron annihilation, which is particularly important for the determination of TMD \textcolor{red}{FFs~\cite{Bacchetta:2015ora}}. This work can therefore be considered as the first attempt at a global fit of TMDs. 

\textcolor{red}{The paper is organized as follows. In Sec.~\ref{s:theory}, the general formalism for TMDs in SIDIS and DY processes is briefly outlined, including a description of the assumptions and approximations in the phenomenological implementation of TMD evolution equations. In Sec.~\ref{s:data}, the criteria for selecting the data analyzed in the fit are summarized and commented. In Sec.~\ref{s:results}, the results of our global fit are presented and discussed. In Sec.~\ref{s:end}, we draw some conclusions. }
   

%%%%%%%%%%%%%%%%%%%%%%%%%%%%%%%%%%%%%%%%%%%%%%%%%%%%%%%%%%%%%%%%%%
\section{Formalism}
\label{s:theory}
%%%%%%%%%%%%%%%%%%%%%%%%%%%%%%%%%%%%%%%%%%%%%%%%%%%%%%%%%%%%%%%%%%

\textcolor{blue}{Shall we add pictures for the kinematics of SIDIS and DY data ? E.g. see Fig. 1 in~\cite{Signori:2013mda}.}

%%%%%%%%%%%%%%%%%%%%%%%%%%%%%%%%%%%%
\subsection{Semi-inclusive DIS}
\label{ss:SIDIS_formalism}

In one-particle \textcolor{red}{SIDIS}, a lepton $\ell$ with momentum $l$ scatters 
off a hadron target $N$ with mass $M$ and momentum
$P$. In the final state, the scattered lepton \textcolor{red}{momentum} $l'$ is measured together with
one hadron $h$ with mass $M_h$
and momentum $P_h$. The corresponding reaction formula is  
\begin{equation}
  \label{e:sidis}
\ell(l) + N(P) \to \ell(l') + h(P_h) + X \, .
\end{equation}
The space-like momentum transfer is $q = l - l'$, with $Q^2 = - q^2$. We
introduce the usual invariants  
\begin{align}
  \label{e:xyz}
x &= \frac{Q^2}{2\,P\cdot q},
&
y &= \frac{P \cdot q}{P \cdot l},
&
z &= \frac{P \cdot P_h}{P\cdot q},
&
\gamma &= \frac{2 M x}{Q} .
\end{align}

The available data refer to \textcolor{red}{SIDIS} hadron multiplicities, namely to the differential number of hadrons produced per corresponding inclusive DIS event. In terms of cross sections, we define the multiplicities as
\begin{equation}
m_N^h (x,z,|\bm{P}_{h\Tperp}|, Q^2) = \frac{d \sigma_N^h / ( dx  dz d|\bm{P}_{h\Tperp}| dQ^2) }
                                                                   {d\sigma_{\text{DIS}} / ( dx dQ^2 ) }\, ,
\label{e:multiplicity}
\end{equation}
where $d\sigma_N^h$ is the differential cross section for the \textcolor{red}{SIDIS} process and $d\sigma_{\text{DIS}}$ is the corresponding inclusive one, 
and where \( \bm{P}_{h\Tperp} \) is the component of \( \bm{P}_{h} \) transverse to \( \bm{q} \). 
In the single-photon-exchange approximation, the multiplicities can be written as ratios of
structure functions (see \cite{Bacchetta:2006tn} for details):
\begin{equation}
m_N^h (x,z,|\bm{P}_{h\Tperp}|, Q^2) =   
\frac{2 \pi\,|\bm{P}_{h\Tperp}| F_{UU ,T}(x,z,\bm{P}_{h\Tperp}^2, Q^2) + 2 \pi
  \varepsilon |\bm{P}_{h\Tperp}| F_{UU ,L}(x,z,\bm{P}_{h\Tperp}^2, Q^2)}
        {F_{T}(x,Q^2) + \varepsilon  F_{L}(x,Q^2)} \, ,
 \label{e:mFF}
\end{equation} 
where
\begin{align}
\varepsilon &= \frac{1-y -\frac{1}{4} \gamma^2 y^2}{1-y+\frac{1}{2} y^2 +\frac{1}{4} \gamma^2 y^2} \ ,
\end{align}  
\textcolor{red}{and the structure function $F_{XY,Z}$ corresponds to a lepton with polarization $X$ scattering on a target with polarization $Y$ by exchanging a virtual photon in a polarization state $Z$.}

\textcolor{red}{The semi-inclusive cross section can be expressed in a factorized form in terms of TMDs only in the kinematical limits $M^2 \ll Q^2$ and $\bm{P}_T^2 \ll Q^2$. In these limits, the structure function $F_{UU,L}$ of Eq.~\eqref{e:mFF} can be neglected~\cite{Bacchetta:2008xw}. The structure function $F_L$ in the denominator contains contributions involving powers of the strong coupling constant $\alpha_S$ at an order that goes beyond the level reached in this analysis; hence, it will be consistently neglected (see also Ref.~\cite{Signori:2013mda}). }

To express the structure functions in terms of TMD \textcolor{red}{distribution and fragmentation functions}, 
we rely on the factorized formula 
for \textcolor{red}{SIDIS} at low transverse  
momenta~\cite{Collins:1981uk,Collins:1984kg,Ji:2002aa,Ji:2004wu,%
Collins:2011zzd,Aybat:2011zv,GarciaEchevarria:2011rb,Echevarria:2012pw,%
Collins:2012uy}:  
\begin{align}
\label{e:SIDISkT}
   F_{UU,T}(x,z, \bm{P}_{h \Tperp}^2, Q^2) &= \sum_a \mathcal{H}_{UU,T}^{a}(Q^2;\mu^2) \\ 
      &\times \int d\bm{k}_\T^{} \, d\bm{P}_\T^{} \,  f_1^a\big(x,\bm{k}_{\T}^2; \mu^2 \big) \, D_{1}^{a\smarrow h}\big(z,\bm{P}_{\T}^2; \mu^2 \big) \,
      \delta \big(z {\bm k}_{\T} - {\bm P}_{h \Tperp} + {\bm P}_{\T}\big)
\nonumber\\&
\nonumber + Y_{UU,T}\big(Q^2, \bm{P}_{h\Tperp}^2\big) + \mathcal{O}\big(M^2/Q^2\big) \, .
\end{align} 
Here, $\mathcal{H}_{UU,T}$ is the hard scattering part; $f_1^a(x,\bm{k}_{\T}^2;
\mu^2)$ is the TMD \textcolor{red}{distribution of unpolarized partons with} flavor $a$ in an unpolarized
proton, carrying longitudinal momentum fraction $x$ and transverse momentum
$\bm{k}_\T$ at the factorization scale $\mu^2$, which in the following we
choose to be equal to $Q^2$.  The $D_1^{a\smarrow h}(z, \bm{P}_{\T}^2;
\mu^2)$ is the \textcolor{red}{function describing the fragmentation of} an unpolarized parton with flavor $a$ into
an unpolarized hadron $h$ carrying longitudinal momentum fraction $z$ and
transverse momentum 
$\bm{P}_\T$. The term $Y_{UU,T}$ is introduced to ensure a matching
to the perturbative \textcolor{red}{fixed-order} calculations at \textcolor{red}{higher transverse momenta.} 

\textcolor{blue}{Specific challenges related to the application of the TMD formalism to SIDIS at low $Q$ have been recently pointed out~\cite{Collins:2016hqq}. In our work, we leave a detailed treatment of the matching to the high $P_{hT} \approx Q$ region to future investigations. }
\textcolor{red}{Here, because of the above kinematical limits the $Y_{UU,T}$ term and corrections from higher twists of order $M^2/Q^2$ or higher can be neglected. Moreover, in this analysis we resum the soft gluon radiation up to the Next-to-Leading-Log level (NLL). Consistently, the hard scattering part is computed at leading order in $\alpha_S$, namely $\mathcal{H}_{UU,T} (Q^2, \mu^2) \approx 1$. }

To the purpose of applying TMD evolution equations, 
\textcolor{red}{we need to calculate the Fourier transform of the the part of Eq.~\eqref{e:SIDISkT} involving TMDs. The structure function thus reduces to} 
\begin{align}
\label{e:SIDISkTFF}
   F_{UU,T}(x,z, \bm{P}_{h \Tperp}^2, Q^2) &\approx \sum_a  
      \int_0^{\infty} \frac{d b_T}{2 \pi} b_T J_0\big(b_T |\bm{P}_{hT}|/z\big)
      \tilde{f}_1^a\big(x, b_T;\mu^2\big) \tilde{D}_1^{a\smarrow h}\big(z, b_T; \mu^2 \big)
\nonumber \, .
\end{align} 

\textcolor{red}{ E' questa la formula usata nel codice, o serve dire altro?}


%%%%%%%%%%%%%%%%%%%%%%%%%%%%%%%%%%%%
\subsection{Drell--Yan processes}
\label{ss:DY_formalism}

In a Drell--Yan process, two hadrons $A$ and $B$ with momenta $P_A$ and $P_B$ collide \textcolor{red}{at a center-of-mass energy squared $s = (P_A + P_B)^2$} and produce a virtual photon or \textcolor{red}{a} $Z$ boson plus hadrons. 
The boson decays into a
lepton-antilepton pair. The reaction formula is
\begin{equation}
A(P_A)+B(P_B)\to [\gamma^*/Z + X \to] \ell^+(l) + \ell^-(l') + X.
\end{equation} 
The invariant mass of the virtual photon is $Q^2=q^2$ with $q = l + l'$. 
We introduce the rapidity of the virtual photon
\begin{equation}
\eta=\frac{1}{2}\log\bigg(\frac{E+q_z}{E-q_z}\bigg)\  .
\end{equation} 
\textcolor{red}{Io preferisco la versione con $q^0$ oppure $\nu$ al posto di $E$, che potrebbe venir confusa con l'energia dello stato adronico iniziale, cio\`e $E=P^0$..} 
The cross section can be written in terms of structure
functions~\cite{Boer:2006eq,Arnold:2008kf}. For our purposes, we need the unpolarized 
cross section
integrated over $d\Omega$ and over the azimuthal angle of the virtual photon, 
\begin{align}
\frac{d\sigma}{dQ^2\, dq_T^2\,d\eta} &= \sigma_0^{\gamma,Z}
\bigg(F_{UU}^1 + \frac{1}{2} F_{UU}^2\bigg). 
\end{align} 
The elementary cross sections are
\begin{align}
\sigma_0^{\gamma} &= \frac{4\pi \alpha^2_{\rm em}}{3 Q^2 s},
&
\sigma_0^Z &= 
%\frac{\sqrt{2} \pi G_F M_Z^2}{s}
\frac{\pi^2 \alpha_{\rm em}}{s (\sin^2{\theta_W} \cos^2{\theta_W}}
B_R(Z\rightarrow \ell^+\ell^-)
\delta(Q^2 - M_Z^2), 
\end{align} 
where $\theta_W$ is Weinberg's angle, $M_Z$ is the mass of the $Z$ boson \textcolor{red}{, and $B_R$ is the branching ratio for the $Z$ boson decay in two leptons}.
We adopted the narrow-width approximation \textcolor{red}{(che cosa vuol dire?)}. We used the values 
$\sin^2 \theta_W= 0.2313$, $M_Z = 91.18$ GeV \textcolor{red}{, and} 
$B_R(Z\rightarrow \ell^+\ell^-)=3.366$.  

\textcolor{red}{[ Secondo me nella prima delle eq.(10) ci vuole $\pi^2$. Infatti, partendo da Ref.[14] abbiamo}
\begin{align}
\frac{d\sigma}{d^4q} &= \frac{\alpha^2}{s Q^2}\, \frac{8 \pi}{3} \, \left[ F_{UU}^1 + \frac{1}{2}\, F_{UU}^2 \right] 
\nonumber
\end{align}
\textcolor{red}{Ora $d^4 q = dq^+ dq^- dq_x dq_y$. Lo Jacobiano della trasformazione $|| dQ^2 d\eta / dq^+ dq^- || = 2$. Mentre quello per $||dq_T^2 d\theta_q / dq_x dq_y|| = 2$. Quindi }
\begin{align}
\frac{d\sigma}{dQ^2 d\eta dq_T^2 d\theta_q} &= \frac{1}{4} \, \frac{d\sigma}{d^4q} = \frac{\alpha^2}{s Q^2}\, \frac{2 \pi}{3} \, \left[ F_{UU}^1 + \frac{1}{2}\, F_{UU}^2 \right] 
\nonumber
\end{align}
\textcolor{red}{L'ulteriore integrazione in $d\theta_q$ fornisce un $2\pi$, quindi}
\begin{align}
\frac{d\sigma}{dQ^2 d\eta dq_T^2} &= \frac{\alpha^2}{s Q^2}\, \frac{4 \pi^2}{3} \, \left[ F_{UU}^1 + \frac{1}{2}\, F_{UU}^2 \right] 
\nonumber
\end{align}
\textcolor{red}{Vi torna? ]}

\textcolor{red}{Similarly to the SIDIS case, in the kinematical limit $q_T^2 \ll Q^2$ and neglecting the hadron masses the structure function $F_{UU}^2$ can be neglected.}
%Similar reasons as for the semi-inclusive DIS case leads us to neglecting
%the structure function $F_{UU}^2$.

The longitudinal momentum fractions can be written in terms of
rapidity in the following way 
\begin{align}
x_A &= \frac{Q}{\sqrt{s}} e^{\eta},
&
x_B &= \frac{Q}{\sqrt{s}} e^{-\eta}.
\label{xab}
\end{align} 
\textcolor{red}{[ ho cambiato $y$ in $\eta$ per consistenza ]}
Some experiments use the variable $x_F$, which is connected to the other
variables  by the following relations
\begin{align}
\eta &= \sinh^{-1}\bigg(\frac{\sqrt{s}}{Q}\frac{x_F}{2}\bigg),
& 
x_{A} &= \sqrt{\frac{Q^2}{s} + \frac{x_F^2}{4}} + \frac{x_F}{2},
&
x_B &= x_A - x_F.  
\end{align} 

The structure function $F_{UU}^1$ can be written as
\begin{align}
\label{e:DYkT}
   F_{UU}^1(x_A,x_B, \bm{q}_{T}^2, Q^2) &= \sum_a \mathcal{H}_{UU}^{1 a}(Q^2;\mu^2) \\ 
      &\times \int d\bm{k}_{\T A}^{} \, d\bm{k}_{\T B}^{} 
\,  f_1^a\big(x_A,\bm{k}_{\T A}^2; \mu^2 \big) 
\, f_{1}^{\bar{a}}\big(x_B,\bm{k}_{\T B}^2; \mu^2 \big) \,
      \delta \big({\bm k}_{\T A} - {\bm q}_T + {\bm k}_{\T B}\big)
\nonumber\\&
\nonumber + Y_{UU}^1\big(Q^2, \bm{q}_T^2\big) + \mathcal{O}\big(M^2/Q^2\big) \, .
\end{align} 
\textcolor{red}{[ ho cambiato $\bm{q}_{\T}$ in $\bm{q}_T$ per consistenza ]}

\textcolor{red}{As in the SIDIS case, with the above kinematical limits the $Y_{UU}$ term and corrections from higher twists of order $M^2/Q^2$ or higher can be neglected. Consistently with our NLL analysis, the hard coefficients become }
\begin{align} 
{\cal H}_{UU, \gamma}^{1 a}(Q^2;\mu^2) &\approx \frac{e_a^2}{N_c},
&
{\cal H}_{UU, Z}^{1 a}(Q^2;\mu^2) &\approx \frac{V_a^2+A_a^2}{N_c} \ ,
\end{align}  
where\footnote{We remind the reader that the value of weak isospin $I_3$ is equal to $+1$ for $u$, $c$, $t$ and
  $-1$ for $d$, $s$, $b$.}
\begin{align}
V_a & = I_{3a} - 2 e_{a} \sin \theta_W \  ,
&
A_a & = I_{3a} \  .
\end{align} 

\textcolor{red}{The structure function can be conveniently expressed as a Fourier transform of the right-handside of Eq.~\eqref{e:DYkT} as }
\begin{align}
\label{e:DYkTFF}
   F_{UU}^1(x_A,x_B, \bm{q}_T^2, Q^2) &=
 \sum_a {\cal H}_{UU}^{1 a} \, \int_0^{\infty} \frac{d b_T}{2 \pi} b_T\, J_0\big( b_T |\bm{q}_T|\big)\ 
      \tilde{f}_1^a\big(x_A, b_T;\mu^2\big) \   \tilde{f}_1^{\bar{a}}\big(x_B, b_T;\mu^2 \big)  \, .
\end{align} 

\textcolor{red}{ Stessa osservazione che in SIDIS: \`e questa la formula usata nel codice, o serve dire altro?}



%%%%%%%%%%%%%%%%%%%%%%%%%%%%%%%%%%%%
\subsection{TMDs and their evolution}
\label{ss:TMDevo}

\textcolor{red}{Following the CSS formalism of Refs.~\cite{Collins:2011zzd,Aybat:2011zv}, the unpolarized TMD distribution and fragmentation functions in configuration space for a parton flavor $a$ at a certain scale $\mu^2$ can be represented as}
\begin{align}   
\widetilde{f}_1^a (x,  b_T; \mu^2) &= \sum_{i=q,\bar q,g} \bigl( C_{a/i} \otimes f_1^i \bigr) (x, \bar{b}_{\ast}; \mu_b^2) 
\  e^{S (\bar{b}_{\ast}; \mu_b^2, \mu^2)} \  e^{g_K(b_T) \ln (\mu^2 / Q_0^2)} \  \widetilde{f}_{1 {\rm NP}}^a (x, b_T) \ ,
\label{e:TMDevol1} \\
\widetilde{D}_1^{a\to h} (z, b_T; \mu^2) &= \sum_{i=q,\bar q,g} \bigl( \hat{C}_{a/i} \otimes D_1^{i\to h} \bigr) (z, \bar{b}_{\ast}; \mu_b^2) \  e^{\hat{S} (\bar{b}_{\ast}; \mu_b^2, \mu^2)} \  e^{g_K( b_T) \ln (\mu^2 / Q_0^2)} \  \widetilde{D}_{1 {\rm NP}}^{a\to h} (z, b_T) \  .
\label{e:TMDevol2}
\end{align}
\textcolor{red}{The $C$ and $\hat{C}$ are perturbatively calculable Wilson coefficients for the TMD distribution and fragmentation functions, respectively. They are convoluted with the corresponding collinear functions as}
\begin{align}
\bigl( C_{a/i} \otimes f_1^i \bigr) (x, \bar{b}_{\ast}; \mu_b^2) &= \int_x^1 \frac{du}{u}\  C_{a/i} \left( \frac{x}{u}, \bar{b}_{\ast}; \mu_b^2 \right) \  f_1^i (u; \mu_b^2) \  , 
\label{e:WC1} \\
\bigl( \hat{C}_{a/i} \otimes D_1^{i\to h} \bigr) (z, \bar{b}_{\ast}; \mu_b^2) &= \int_z^1 \frac{du}{u}\  \hat{C}_{a/i} \left( \frac{z}{u}, \bar{b}_{\ast}; \mu_b^2 \right) \  D_1^{i\to h} (u; \mu_b^2) \  . 
\label{e:WC2}
\end{align}
\textcolor{red}{The convolutions are only valid for small $b_T \ll 1/\Lambda_{\rm QCD}$. At larger $b_T$, the TMDs need to match the nonperturbative expressions $\widetilde{f}_{1 \rm NP}^a$ and $\widetilde{D}_{1 {\rm NP}}^{a\to h}$, respectively, that must be constrained by fitting experimental data. The evolution of TMDs from the initial scale $Q_0$ to $\mu$ is carried out through perturbatively calculable Sudakov factors $S$ and $\hat{S}$, respectively, and through a nonperturbative universal term $g_K$ at large $b_T$ that accounts for the radiation of soft gluons emitted by the considered parton. }

\textcolor{red}{The matching between small (perturbative) and large (nonperturbative) $b_T$ is controlled by the $\mu_b$ scale, which naturally should be proportional to $1/b_T$. We choose }
\begin{align} 
\mu_b &= \frac{2 e^{-\gamma_E}}{\bar{b}_{\ast}} \  ,
\label{e:mub}
\end{align}  
\textcolor{red}{where $\gamma_E$ is the Euler constant and }
\begin{align} 
\bar{b}_{\ast} &\equiv b_{\rm max} \Bigg(\frac{1-e^{- b_T^4 / b_{\rm max}^4} }{1-e^{- b_T^4 / b_{\rm min}^4}} \Bigg)^{1/4} .
\label{e:b*}
\end{align}  
\textcolor{red}{This variable replaces the simple dependence upon $b_T$ in the convolutions of Eqs.~\eqref{e:WC1}, \eqref{e:WC2} and in the perturbative Sudakov factors $S$ and $\hat{S}$; namely, in the perturbative parts of the TMD definitions of Eqs.~\eqref{e:TMDevol1},~\eqref{e:TMDevol2}. In fact, at large $b_T$ these parts are no longer reliable. Therefore, the $\bar{b}_{\ast}$ is chosen to saturate on the maximum value $b_{\rm max}$, as suggested by the CSS formalism~\cite{Collins:2011zzd,Aybat:2011zv}. }\footnote{We remind that different schemes are possible to deal with
the high-$b_T$ region like the so-called ``complex-$b$ prescription''~\cite{Laenen:2000de}.} \textcolor{red}{On the other hand, at small $b_T$ the TMD formalism must match the fixed-order collinear calculations where the $b_T$ dependence is perturbatively generated. The form of the matching is arbitrary. Here, we choose to saturate $\bar{b}_{\ast}$ on the minimum value $b_{\rm min}\propto 1/Q$. In general, both $b_{\rm max}$ and $b_{\rm min}$ must not be considered as free parameters; rather, they should be regarded as arbitrary scales separating perturbative from nonperturbative regimes~\cite{Collins:2014jpa}.} We choose to fix them \textcolor{red}{on} the values
\begin{align}
b_{\rm max} &= 2 e^{-\gamma_E}  \text{  GeV}^{-1} = 1.123 \text{  GeV}^{-1}, 
&
b_{\rm min} &= 2 e^{-\gamma_E}/Q \ .
\label{e:bminmax}
\end{align} 

The motivations are the following: 
\begin{itemize}
\item{} because of the choices~\eqref{e:bminmax}, the scale $\mu_b$ is constrained between 1 GeV and $Q$, so that the collinear PDFs are never computed at a scale lower than 1 GeV and the lower limit of the \textcolor{red}{integrals contained in the definition of} the perturbative Sudakov factor can never \textcolor{red}{become larger} than the upper limit

\item{} at $Q_0 = 1$ GeV, $b_{\rm max} = b_{\rm min}$ and \textcolor{red}{there are no evolution effects;} the TMD is
simply given by the \textcolor{red}{corresponding} collinear \textcolor{red}{function multiplied by} a nonperturbative contribution \textcolor{red}{depending on the intrinsic $b_T$} (plus possible corrections of order $\alpha_S$ from the Wilson coefficients)
 
\item{} Our choice partially corresponds to modifying the resummed \textcolor{red}{leading logarithms in the gluon radiation as in Ref.~\cite{Bozzi:2010xn}.} 
\end{itemize}
\textcolor{red}{By integrating over the impact parameter $b_T$, the collinear expression for both distribution and fragmentation functions can be recovered. }

\textcolor{red}{Following Refs.~\cite{Nadolsky:1999kb,Landry:2002ix,Konychev:2005iy}, for the nonperturbative Sudakov factor we make the traditional choice $g_K (b_T) = - g_2 b_T^2 / 2$ with $g_2$ a free parameter. Recently, several alternative forms have been proposed~\cite{Aidala:2014hva,Collins:2014jpa} including the suggestion of not including such term~\cite{D'Alesio:2014vja}.}

\textcolor{red}{The intrinsic nonperturbative parts of the TMDs are}
\begin{align}
\widetilde{f}_{1 {\rm NP}}^a (x, b_T) &= e^{-\langle \bm{k}_{\T a}^2 \rangle \frac{b_T^2}{4}}
        \bigg( 1 - \frac{\lambda}{1+\lambda} \  \langle \bm{k}_{\T a}^2 \rangle \frac{b_T^2}{4} \bigg)\  ,
\label{e:f1NP} \\
\widetilde{D}_{1 {\rm NP}}^{a \to h} (z, b_T) &= 
    \frac{ \langle \bm{P}_{\T a\to h}^2 \rangle \   e^{-\langle \bm{P}_{\T a\to h}^2 \rangle \frac{b_T^2}{4}}
        + \lambda_F \   \langle \bm{P}_{\T a\to h}'^2 \rangle \left( 1 - \langle \bm{P}_{\T a\to h}'^2 \rangle \frac{b_T^2}{4} \right)
         \  e^{- \langle \bm{P}_{\T a\to h}'^2 \rangle \frac{b_T^2}{4}}}
     {\langle \bm{P}_{\T a\to h}^2 \rangle + \lambda_F\   \langle \bm{P}_{\T a\to h}'^2 \rangle} \  .
\label{e:D1NP}
\end{align} 
\textcolor{red}{After performing the anti-Fourier transform, the $f_{1 \rm NP}$ and $D_{1\rm NP}$ in momentum space look like the normalized linear combination of two different Gaussians:}
\begin{align} 
f_{1 {\rm NP}}^a (x, \bm{k}_{\T}) &= \frac{1}{\pi} \  
                        \frac{\big( 1 +\lambda \bm{k}_{\T}^2\big)}
                                {\langle \bm{k}_{\T a}^2 \rangle +\lambda \  \langle \bm{k}_{\T a}^2 \rangle^2}
                        \  e^{- \frac{\bm{k}_{\T}^2}{\langle \bm{k}_{\T a}^2 \rangle}} \  ,
\label{e:f1NPk}   \\
D_{1 {\rm NP}}^{a\to h} (z, \bm{P}_{\T}) &=  \frac{1}{\pi} \   
                       \frac{1}{\langle \bm{P}_{\T a\to h}^2 \rangle + \lambda_F \  \langle \bm{P}_{\T a\to h}'^2 \rangle^2}
                \   \bigg( e^{- \frac{\bm{P}_{\T}^2}{\langle \bm{P}_{\T a\to h}^2 \rangle}}
                              + \lambda_F \  \bm{P}_{\T}^2 \  e^{- \frac{\bm{P}_{\T}^2}{\langle \bm{P}_{\T a\to h}'^2 \rangle}} \bigg) \  .
\label{e:D1NPk}
\end{align} 

\textcolor{red}{Based on the analyses of Refs.~\cite{Signori:2013mda,Bacchetta:2015ora}, the Gaussian width of the TMD distribution depends on the parton flavor $a$ and on its fractional momentum $x$ according to}
\begin{align} 
\big\langle \bm{k}_{\T a}^2 \big\rangle (x) &= \big\langle \hat{\bm{k}}_{\T a}^2 \big\rangle \;  
\frac{(1-x)^{\alpha} \  x^{\sigma} }{ (1 - \hat{x})^{\alpha} \  \hat{x}^{\sigma} } \, ,
\label{e:kT2_kin}
\end{align}
\textcolor{red}{where $\alpha, \, \sigma,$ and $\big\langle \hat{\bm{k}}_{\T a}^2 \big\rangle \equiv \big\langle \bm{k}_{\T a}^2 \big\rangle (\hat{x})$ with $\hat{x} = 0.1$, are free parameters. Similarly, we have}
\begin{align}  
\big\langle \bm{P}_{\T a \to h}^2 \big\rangle (z) &= \big\langle \hat{\bm{P}}_{\T a \to h}^2 \big\rangle \  
               \frac{ (z^{\beta} + \delta)\ (1-z)^{\gamma} }{ (\hat{z}^{\beta} + \delta)\   (1 - \hat{z})^{\gamma} } \, ,
 \label{e:PT2_kin}
 \end{align}
 \textcolor{red}{where $\beta, \, \gamma, \, \delta, $ and $\big\langle \hat{\bm{P}}_{\T a \to h}^2 \big\rangle \equiv \big \langle \bm{P}_{\T a \to h}^2 \big\rangle (\hat{z})$ with $\hat{z} = 0.5$, are free parameters. For sake of simplicity, the $\beta, \, \gamma, \, \delta$ parameters are taken equal for all fragmentation channels~\cite{Signori:2013mda,Bacchetta:2015ora}. }





\newpage
%%%%%%%%%%%%%%%%%%%%%%%%%%%%%%%%%%%%%%%%%%%%%%%%%%%%%%%%%%%%%%%%%%
\section{Data analysis}
\label{s:data_analysis}
%%%%%%%%%%%%%%%%%%%%%%%%%%%%%%%%%%%%%%%%%%%%%%%%%%%%%%%%%%%%%%%%%%

One of the main goals of our fit is to test the universality of TMD parton distributions and fragmentation functions among different processes.
To achieve this we included measurements taken from semi-inclusive DIS, Drell-Yan and $Z$ boson production from a wide range of experimental collaborations. 
In this chapter we will illustrate the experimental data considered for each process and the reasons behind the kinematic cuts applied to them.

Tab.~\ref{t:data_SIDIS_proton} refers to the experimental data for SIDIS off proton target (\hermes experiment) and presents its kinematic features. 
The same holds for Tab.~\ref{t:data_SIDIS_deuteron}, Tab.~\ref{t:data_DY}, Tab.~\ref{t:data_Z} for SIDIS off deuteron (\hermes and \compass experiments), Drell-Yan events at low energy and $Z$ boson production.


%%%%%%%%%%%%%%%%%%%%%%%%%%%%%%%%%%%%
\subsection{Hermes data}
\label{ss:hermes}

The semi-inclusive DIS data are taken from \textsc{Hermes} collaboration~\cite{Airapetian:2012ki} and \compass experiment \cite{Adolph:2013stb}.
\textcolor{red}{A similar analysis on \textsc{Hermes} data has been alredy done in a previous work~\cite{Signori:2013mda}.} 

\textsc{Hermes} data are grouped in two data sets, distinguished by the inclusion or subtraction of the vector meson contribution. In our work we considered only the vector meson subtracted data set.\\
The collaboration measured the multiplicities for SIDIS in a fixed target experiment using hydrogen and deuteron and separating charged pions and kaons produced in the final state. The data set then includes 8 different channels for every combination of target and final-state hadron for a total of 2688 points.\\
They are divided in bins of $(x,z,Q^2,Ph_T)$ with the average values of $(x,Q^2)$ spanning from about $(0.04, 1.25\text{ GeV}^2)$ to $(0.4, 9.2\text{ GeV}^2)$, while for other variables we have $0.1\leq z\leq 0.9$ and $0.1 \text{GeV} \leq \vert Ph_T \vert \leq 1 \text{ GeV}$.

%%%%%%%%%%%%%%%%%%%%%%%%%%%%%%%%%%%%
\subsection{Compass data}
\label{ss:compass}

Compass collaboration instead extracted multiplicities of charged hadrons produced in SIDIS on a deuteron ($^6\text{LiD}$) target. The data are organised in bins dependent on $(x,z,Q^2,Ph_T)$ as well, however the number of data is an order of magnitude greater than the \textsc{Hermes} ones.\\
The data cover the range of $(x,Q^2)$ from $(0.0052, 1.11\text{GeV}^2)$ to $(0.0932, 7.57\text{GeV}^2)$ and the interval $0.2 \leq z \leq 0.8$.\\
In data sets from both collaborations, for every bin we used the average values for each kinematic variables.\\

To avoid issues relative to errors in the normalization of data, we divided every Compass data point by the value of the first data point of their bin, defining a new variable:
\begin{equation}
m_{\text{norm}} = \frac{m_N^h(x,z,\bm{P}_{h\Tperp}^2, Q^2)}{m_N^h (x,z,{\rm Min}[\bm{P}_{h\Tperp}^2], Q^2)}
\label{eq:mult_norm}
\end{equation}
The first data point of every selected bin  was consequently considered as a fixed parameter and excluded from the degrees of freedom of the system.

\textcolor{red}{The application of the TMD formalism to SIDIS at low energy $Q$ crucially depends on the capability of separating the current from the target fragmentation region and from a soft region. The issue has been recently discussed in~\cite{Boglione:2016bph}. In this paper, }
\textcolor{blue}{we do not implement any combined cut on $z$ and $P_{hT}$. We leave it to future studies\footnote{The implementation of the $R$ cut proposed in~\cite{Boglione:2016bph} crucially depends on the value of $\langle k_T^2 \rangle$, features that requires independent determinations of the quark properties.}}
%\textcolor{blue}{consistently with the way experimental data are presently binned, we identify the current fragmentation region by means of cuts on $z$.}

%%%%%%%%%%%%%%%%%%%%%%%%%%%%%%%%%%%%
\subsection{Low-energy Drell-Yan data}
\label{ss:dy}

In the case of Drell-Yan data we started our analysis on data sets considered in previous works \textcolor{red}{[CITE]}. We used data from E288~\cite{Ito:1980ev} measured at $\sqrt{s}=19.4,\,23.8$ and $27.4\text{ Gev}^2$, denoted with the name 200, 300 and 400 respectively. We included also data from E605~\cite{Moreno:1990sf} at $\sqrt{s}=38.8 \text{ GeV}^2$.

%%%%%%%%%%%%%%%%%%%%%%%%%%%%%%%%%%%%
\subsection{Z-boson production data}
\label{ss:zboson}

We needed also data at higher $q_T$, so we considered also data taken from Z boson production in collider experiments at Tevatron.  We used data from CDF and D0, from Run I~\cite{Affolder:1999jh,Abbott:1999wk} at $\sqrt{s}=1.8\text{ TeV}$ and Run II~\cite{Aaltonen:2012fi,Abazov:2007ac} at $\sqrt{s}=1.96\text{ TeV}$. The invariant mass for this kind of experiments is $M=M_Z$, while the transverse momentum exchanged spans $0< q_T < 20 \text{GeV}$.
The quantity used in the fit for Z boson production data is $d\sigma /dq_T$,  however in the case of D0 Run II the data published contain the quantity $1/\sigma \times d\sigma/dq_T$ so we multiplied every one of this point for the cross section of this process $\sigma_{exp} = 255.8 \pm 16 \text{ pb}$. The errors relative to the cross section and the data published have been added in quadrature.\\

\textcolor{red}{Cuts and reasons}\\
\textcolor{red}{ERRORS}


\renewcommand{\tabcolsep}{0.4pc} % enlarge column spacing
\renewcommand{\arraystretch}{1.3} % enlarge line spacing

%%%%%%%%%%%%%% Tab. SIDIS data off proton %%%%%%%%%%%%%%%%%%%%%%%%%%
\begin{table}[h!]
\begin{center}
\begin{tabular}{|c|c|c|c|c|}
 \hline
  & HERMES & HERMES & HERMES & HERMES \\
 ~          &  $p \to \pi^+$    &   $p \to \pi^-$    &  $p \to K^+$    &   $p \to K^-$               \\
 \hline
 Reference & \multicolumn{4}{c|}{\cite{Airapetian:2012ki}}        \\
\hline
\multirow{3}{*}{Cuts}             & \multicolumn{4}{c|}{$Q^2 > 1.4 \text{ GeV}^2$}     \\
             & \multicolumn{4}{c|}{$0.2 <z <0.7$}     \\
             & \multicolumn{4}{c|}{$P_{h \Tperp}< {\rm Min}[0.2\ Q, 0.7 \ Q z ] +0.5$ GeV}     \\
\hline
 Points         &  188 & 186 & 187 & 185       \\
 \hline
Max. $Q^2$      &  \multicolumn{4}{c|}{$9.2 \text{ GeV}^2 $}               \\
 \hline
$x$ range       & \multicolumn{4}{c|}{$0.06 < x < 0.4$ }                \\
\hline
%Notes         &\multicolumn{4}{c|}{ }   \\ 
%\hline 
%$\chi^2 /$points &4.24 & 3.49 & 0.63 & 1.00                \\
%\hline
\end{tabular}
\caption{Semi-inclusive DIS proton-target data (Hermes experiment).}
\end{center}
\label{t:data_SIDIS_proton}
\end{table}
%%%%%%%%%%%%%% Tab. SIDIS data off deuteron %%%%%%%%%%%%%%%%%%%%%%%%%%
\begin{table}[h!]
\begin{center}
\begin{tabular}{|c|c|c|c|c|c|c|}
 \hline
  & HERMES & HERMES & HERMES & HERMES & COMPASS & COMPASS\\
 ~          &  $D \to \pi^+$    &   $D \to \pi^-$    &  $D \to K^+$    &   $D \to K^-$      &  $D \to h^+$    &   $D \to h^-$            \\
 \hline
 Reference & \multicolumn{4}{c|}{\cite{Airapetian:2012ki}}        &\multicolumn{2}{c|}{\cite{Adolph:2013stb}} \\
\hline
\multirow{3}{*}{Cuts}             & \multicolumn{6}{c|}{$Q^2 > 1.4 \text{ GeV}^2$}     \\
             & \multicolumn{6}{c|}{$0.2 <z <0.7$}     \\
             & \multicolumn{6}{c|}{$P_{h \Tperp}< {\rm Min}[0.2\ Q, 0.7 \ Q z ] +0.5$ GeV}     \\
\hline
 Points         &  188 & 188 & 186 & 187       &      3024    &   3021                 \\
 \hline
Max. $Q^2$      &  \multicolumn{4}{c|}{$9.2 \text{ GeV}^2 $}      & \multicolumn{2}{c|}{$10 \text{ GeV}^2 $}             \\
 \hline
$x$ range       & \multicolumn{4}{c|}{$0.06 < x < 0.4$ }    &  \multicolumn{2}{c|}{$0.006 < x < 0.12$ }             \\
\hline
Notes         &\multicolumn{4}{c|}{ }   & \multicolumn{2}{c|}{Observable: $\displaystyle \frac{m_N^h
    (x,z,\bm{P}_{h\Tperp}^2, Q^2)}{m_N^h (x,z,{\rm Min}[\bm{P}_{h\Tperp}^2], Q^2)}$}             \\
\hline 
%$\chi^2 /$points &3.06 & 2.53 & 1.04 & 3.18    &  1.48        &    0.96             \\
%\hline
\end{tabular}
\caption{Semi-inclusive DIS deuteron-target data (Hermes and Compass experiments).}
\end{center}
\label{t:data_SIDIS_deuteron}
\end{table}
%%%%%%%%%%%%%% Tab. Drell-Yan data low Q %%%%%%%%%%%%%%%%%%%%%%%%%%
\begin{table}[h!]
\begin{center}
%\newcommand{\m}{\hphantom{$-$}}
%\newcommand{\cc}[1]{\multicolumn{1}{c}{#1}}
\renewcommand{\tabcolsep}{0.4pc} % enlarge column spacing
\renewcommand{\arraystretch}{1.2} % enlarge line spacing
\begin{tabular}{|c|c|c|c|c|}
 \hline
 ~                        &  E288 200    &  E288 300        &  E288 400          &  E605                \\
 \hline
Reference               &  \cite{Ito:1980ev}  &   \cite{Ito:1980ev}  &  \cite{Ito:1980ev}  &   \cite{Moreno:1990sf}  \\
\hline
Cuts             & \multicolumn{4}{c|}{$q_T < 0.2\ Q +0.5$ GeV}
\\
 \hline
 Points                   &      45      &   45             &       78           &     35               \\
 \hline
 $\sqrt{s}$               &    19.4 GeV   &   23.8 GeV        &      27.4 GeV    &  38.8 GeV           \\
\hline
$Q$ range                 &  4-9 GeV      &  4-9 GeV         &  5-9, 11-14 GeV   &  7-9, 10.5-18 GeV   \\
 \hline
 Kin. var.           & $y$=0.4         &  $y$=0.21          &   $y$=0.03         &    $-0.1<x_F< 0.2$         \\
\hline
%$ \chi^2  /$points      &  0.52        &    0.98           &       0.68         &   0.68     \\
%\hline
\end{tabular}
\caption{Low energy Drell-Yan data collected by the E288 and E605 experiments at Tevatron, with different center-of-mass energies.}
\end{center}
\label{t:data_DY}
\end{table}
%%%%%%%%%%%%%% Tab. Z data Tevatron %%%%%%%%%%%%%%%%%%%%%%%%%%
\begin{table}[h!]
\begin{center}
%\vskip 18pt
%\newcommand{\m}{\hphantom{$-$}}
%\newcommand{\cc}[1]{\multicolumn{1}{c}{#1}}
\renewcommand{\tabcolsep}{0.4pc} % enlarge column spacing
\renewcommand{\arraystretch}{1.2} % enlarge line spacing
\begin{tabular}{|c|c|c|c|c|}
 \hline
 ~                        & CDF Run I    &  D0 Run I        & CDF Run II        & D0 Run II      \\
 \hline
 Reference        &\cite{Affolder:1999jh} &\cite{Abbott:1999wk}&\cite{Aaltonen:2012fi}&\cite{Abazov:2007ac} \\
\hline
Cuts             & \multicolumn{4}{c|}{$q_T< 0.2\ Q +0.5 \text{ GeV}=18.7$ GeV}                                  \\
\hline
 Points                   &      31      &   14             &       37          &        8       \\
 \hline
 $\sqrt{s}$               &      1.8 TeV &   1.8 TeV        &       1.96 TeV    &       1.96 TeV   \\
 \hline
Normalization        &  1.114       &    0.992          &       1.049        &       1.048    \\
\hline
%$\chi^2 /$points      &  0.52        &    0.98           &       0.68         &   0.68     \\
%\hline
\end{tabular}
\caption{$Z$ boson production data collected by the CDF and D0 experiments at Tevatron, with different center-of-mass energies. \textcolor{blue}{Discuss the meaning of the normalization factors}.}
\end{center}
\label{t:data_Z}
\end{table}
%%%%%%%%%%%%%%%%%%%%%%%%%%%%%%%%%%%%%%%%%%%%%%%%%%%%%%%%%%%%%%%



%%%%%%%%%%%%%%%%%%%%%%%%%%%%%%%%%%%%
\subsection{The replica method}
\label{ss:replica_method}

Description of the replica method and the definition of the $\chi^2$ (error function).

\textcolor{blue}{The following text is copy-pasted from~\cite{Signori:2013mda}. We need to edit/rewrite it. Explain: which sources of theoretical errors are we considering in the current fit, apart from the error on the collinear fragmentation functions?}

The fit and the error analysis were carried out using a similar Monte Carlo approach as in Ref.~\cite{Bacchetta:2012ty}, and taking
inspiration from the work of the NNPDF collaboration (see, e.g.,~\cite{Forte:2002fg,Ball:2008by,Ball:2010de}). The approach consists in creating $\mathcal{M}$ replicas of the data points. In each replica (denoted
by the index $r$), each data point $i$ is shifted by a Gaussian noise with the
same variance as the measurement. Each replica, therefore, represents a
possible outcome of an independent experimental measurement, which we denote
by $m_{N, r}^{h}(x, z, \bm{P}_{h\Tperp}^2, Q^2)$. The number of replicas is
chosen so that the mean and standard deviation of the set of replicas
accurately reproduces the original data points. In our case, we have found
that 200 replicas are more than sufficient.

The standard minimization procedure is applied to each replica separately, by
minimizing the following error  
function~\footnote{Note that the error for each replica is taken to be equal
  to the error on the original data points. This is consistent with the fact
  that the variance of the $\mathcal{M}$ replicas should reproduce the
  variance of the original data points.}  
\begin{equation}
%\small
E_r^2(\{p\})=\sum_{i} 
\frac{\Bigl(m_{N, r}^{h}(x_i, z_i, \bm{P}_{h\Tperp i}^2, Q_i^2) - m_{N,  \mbox{\tiny theo}}^{h}(x_i, z_i, \bm{P}_{h\Tperp i}^2; \{p\})\Bigr)^2}
        {\Bigl( \Delta m_{N, \mbox{\tiny stat}}^{h\ 2} + \Delta m_{N, \mbox{\tiny sys}}^{h\ 2} \Bigr)(x_i, z_i, \bm{P}_{h\Tperp i}^2, Q^2_i) +\Bigl(\Delta m_{N, \mbox{\tiny theo}}^{h}(x_i, z_i, \bm{P}_{h\Tperp i}^2) \Bigr)^2}   . 
\label{e:MC_chi2}
\end{equation}
The sum runs over the $i$ experimental points, including all species of
targets $N$ and final-state hadrons $h$. The theoretical multiplicities $m_{N,
  \text{theo}}^{h}$ and their error $\Delta m_{N,  {\rm theo}}^{h}$ do not
depend on $Q^2$, as explained in the previous section. They are computed at
the fixed value $Q^2=2.4$ GeV$^2$ using the formula in
Eq.~\eqref{e:FDmult}. However, in each $z$ bin for each replica the value of
$D_1^{a \smarrow h}$ is independently modified with a Gaussian noise with
standard deviation equal to the theoretical error $\Delta D_1^{a\smarrow
  h}$. The latter is estimated from the plots in Ref.~\cite{Epele:2012vg} and
it represents the main source of uncertainty in $\Delta m_{N,  {\rm
    theo}}^{h}$. Finally, the symbol $\{p\}$ denotes the vector of fitting
parameters.  

The minimization was carried out using the \minuit  code. The final outcome is a set of $\mathcal{M}$ different vectors of best-fit parameters, $\{ p_{0r}\},\; r=1,\ldots \mathcal{M}$, with which we can calculate any observable, its mean, and its standard deviation. 
The distribution of these values needs not to be necessarily Gaussian. In this
case, the $1 \sigma$ confidence interval is different from the 68\%
interval. The 68\% confidence interval can simply be computed for
each experimental point by rejecting the largest and the lowest 16\% of the
$\mathcal{M}$ values.   

Although the minimization is performed on the function defined in Eq.~\eqref{e:MC_chi2}, the agreement of the $\mathcal{M}$ replicas with the original data is better expressed in terms of a $\chi^2$ function defined as in Eq.~\eqref{e:MC_chi2} but with the replacement $m_{N, r}^{h} \to m_{N}^{h}$, i.e.,  with respect to the original data set.
If the model is able to give a good description of the data, the distribution of the $\mathcal{M}$ values of $\chi^2$/d.o.f. should be peaked around one. 
%In practice, the rigidity of our functional form leads to higher $\chi^2$ values. 










\newpage
%%%%%%%%%%%%%%%%%%%%%%%%%%%%%%%%%%%%%%%%%%%%%%%%%%%%%%%%%%%%%%%%%%
\section{Results}
\label{s:results}
%%%%%%%%%%%%%%%%%%%%%%%%%%%%%%%%%%%%%%%%%%%%%%%%%%%%%%%%%%%%%%%%%%

In the following we detail the results of fits to the data sets presented in Sec.~\ref{s:data_analysis}. In Sec.~\ref{ss:fl_ind_fit} we present a fit of TMDs with flavor-independent widths (see Eqs.~\eqref{e:f1NP} and~\eqref{e:D1NP}); in Sec.~\ref{ss:fl_dep_fit} we discuss the flavor-dependent case.


%%%%%%%%%%%%%%%%%%%%%%%%%%%%%%%%%%%%%%%%%%%%%%%%%%%%%%%%%%%%%%%%%%
%%		 	   FLAVOR INDEPENDENT CASE
%%%%%%%%%%%%%%%%%%%%%%%%%%%%%%%%%%%%%%%%%%%%%%%%%%%%%%%%%%%%%%%%%%
\subsection{Flavor independent fit}
\label{ss:fl_ind_fit}

In Tab.~\ref{t:fl_ind_chi2} we present the total $\chi^2$ and its breakdown (division) between SIDIS (separating Hermes and Compass) and Drell-Yan/$Z$ events. \textcolor{blue}{$\chi^2$ values need to be added.} 
Tab.~\ref{t:fl_ind_parcommon} summarizes the values of the nonperturbative parameters $b_{\rm min}$ and $b_{\rm max}$ (which delimit the range in $b_T$ where transverse momentum resummation is computed perturbatively) and $g_2$, whcih quantifies the amount of soft gluons radiated.
Tab.~\ref{t:fl_ind_par_TMD} collects the best-fit values for the quantities used to parametrize the nonperturbative part of the TMDs (Eqs.~\eqref{e:f1NP} and~\eqref{e:D1NP}); central values and standard deviations are based on the replica methodology (see Sec.~\ref{ss:replica_method}), using $68\%$ confidence levels.

%%%%%%%%%%%%%%%% Tab. # SIDIS points %%%%%%%%%%%%%%%%%%%%%%%%%%
\begin{table}[h!]
\small
  \centering
  \begin{tabular}{|c|c|c|c|c|c|c|c|c|c|}
\hline
\hline
%  \multicolumn{4}{|c|}{Parameters for TMD PDFs} \\
%  \hline
%  \hline
Points& Parameters & $\chi^2$& $\chi^2/$d.o.f.& 
                  Points &$\chi^2$& Points &$\chi^2$& Points &$\chi^2$ 
 \\ 
      &    &    &  & HERMES    & HERMES   & COMPASS & COMPASS & DY \& Z & DY \& Z  \\
\hline
8156 & 11  & $12629 \pm 363$ & $1.55 \pm 0.05$ & 1737&  &6126 & & 293 &    \\
\hline
\hline
\end{tabular}
\caption{Flavor-independent fit: number of points analyzed and $\chi^2$ values for SIDIS and Drell-Yan/$Z$ production.}
\label{t:fl_ind_chi2}
\end{table}
%%%%%%%%%%%%%%%% Tab. common parameters %%%%%%%%%%%%%%%%%%%%%%%%%%
\begin{table}[h!]
\small
  \centering
  \begin{tabular}{|c|c|c|}
\hline
\hline
%  \multicolumn{4}{|c|}{Parameters for TMD PDFs} \\
%  \hline
%  \hline
$b_{\rm max}$ & $b_{\rm min}$ &  $g_2$ 
 \\ 
 (fixed)     & (fixed)   & {[GeV$^2$]}                           \\
\hline
$2 e^{-\gamma_E}/$GeV& $2 e^{-\gamma_E}/Q$  & $0.12 \pm 0.01$  \\
\hline
\hline
\end{tabular}
\caption{Flavor-independent fit: values of parameters common to TMD PDFs and TMD FFs.}
\label{t:fl_ind_parcommon}
\end{table}
%%%%%%%%%%%%%%% Tab. results paramters TMDs %%%%%%%%%%%%%%%%%%%%%%
\begin{table}[h!]
\small
  \centering
  \begin{tabular}{|c||c|c|c|c|c|c|}
\hline
\hline
%  \multicolumn{4}{|c|}{Parameters for TMD PDFs} \\
%  \hline
%  \hline
TMD PDFs&  $\big \langle \hat{\bm{k}}_{\T}^2 \big \rangle$ 
& $\alpha$ & $\sigma$ & & $\lambda$ &  
 \\ 
        & {[GeV$^2$]}                               &
      (random) &      &  & & \\
\hline
  & $0.31 \pm 0.08$ & $2.93 \pm 0.07 $ & $0.20 \pm
0.01$  &  & $1.49 \pm 0.96$ &    \\
\hline
\hline
TMD FFs&  $\big \langle \hat{\bm{P}}_{\perp}^2 \big \rangle$ &
$\beta$ & $\delta$ & $\gamma$ & $\lambda_F$ & $\big \langle
\hat{\bm{P}}_{\perp}^{\prime 2} \big \rangle$
 \\ 
        & {[GeV$^2$]} &            &         & & &{[GeV$^2$]}    \\
\hline
   &  $0.20 \pm 0.01$ & $2.7 \pm 0.1 $ & $3.4
\pm 0.1$ & $0.041 \pm 0.004$ & $4.9 \pm 1.2$ & $0.040 \pm 0.001$  \\
\hline
\hline
\end{tabular}
\caption{Flavor-independent fit: $68\%$ confidence intervals of best-fit parameters for TMD PDFs and TMD FFs.}
\label{t:fl_ind_par_TMD}
\end{table}
%%%%%%%%%%%%%%%%%%%%%%%%%%%%%%%%%%%%%%%%%%%%%%%%%%%%%%%%%%%



%%%%%%%%%%%%%%%%%%%%%%%%%%%%%%%%%%%%
\subsubsection{Average transverse momenta}
\label{sss:kT2_PT2}


%%%%%%%%%%%%%%%%%%%%%%%%%%%%%%%%%%%%
\subsubsection{Kinematic dependence}
\label{sss:kindep}

Average square transverse momenta and their kinematic dependence.  (version from Dropbox, Feb. 9$^{th}$).
Full sets of replicas and $68\%$ confidence level bands compared to the results from other fits.
Fix some of the labels on the vertical axis.

%%%%%%%%%%%%%%%%%%%%%%%%%%%%%%%%%
\begin{figure}[h!]
\centering
\begin{tabular}{ccc}
\includegraphics[width=0.40\textwidth]{plots/Parameters_and_Curves/kT2av_curves_at1GeV_flINDEP.pdf}
&\hspace{0.001cm}
&
\includegraphics[width=0.40\textwidth]{plots/Parameters_and_Curves/PT2av_curves_at1GeV_flINDEP.pdf}
\\
(a) && (b)
\end{tabular}
\caption{write the caption here (a) and another here (b).}
\label{f:avmomenta_all_rep}
\end{figure}
%%%%%%%%%%%%%%%%%%%%%%%%%%%%%%%%%
\begin{figure}[h!]
\centering
\begin{tabular}{ccc}
\includegraphics[width=0.40\textwidth]{plots/Parameters_and_Curves/kT2av_Compare_with_other_extractions_flINDEP.pdf}
&\hspace{0.001cm}
&
\includegraphics[width=0.40\textwidth]{plots/Parameters_and_Curves/PT2av_Compare_with_other_extractions_flINDEP.pdf}
\\
(a) && (b)
\end{tabular}
\caption{write the caption here (a) and another here (b).}
\label{f:avmomenta_68CL}
\end{figure}
%%%%%%%%%%%%%%%%%%%%%%%%%%%%%%%%%








%%%%%%%%%%%%%%%%%%%%%%%%%%%%%%%%%%%%%%%%%%%%%%%%%%%%%%%%%%%%%%%%%%
%%		 	   FLAVOR DEPENDENT CASE
%%%%%%%%%%%%%%%%%%%%%%%%%%%%%%%%%%%%%%%%%%%%%%%%%%%%%%%%%%%%%%%%%%
\subsection{Flavor dependent fit}
\label{ss:fl_dep_fit}

\textcolor{blue}{We'll see what we want to discuss here.}
In Tab.~\ref{t:fl_dep_chi2} Tab.~\ref{t:fl_dep_parcommon} Tab.~\ref{t:fl_dep_par_TMD}.

%%%%%%%%%%%%%%%% Tab. # SIDIS points %%%%%%%%%%%%%%%%%%%%%%%%%%
\begin{table}[h!]
\small
  \centering
  \begin{tabular}{|c|c|c|c|c|c|c|c|c|c|}
\hline
\hline
%  \multicolumn{4}{|c|}{Parameters for TMD PDFs} \\
%  \hline
%  \hline
Points& Parameters & $\chi^2$& $\chi^2/$d.o.f.& 
                  Points &$\chi^2$& Points &$\chi^2$& Points &$\chi^2$ 
 \\ 
      &    &    &  & HERMES    & HERMES   & COMPASS & COMPASS & DY \& Z & DY \& Z  \\
\hline
8156 & 18  & $10456 \pm  $ & $1.28 \pm  $ & 1737&  &6126 & & 293 &    \\
\hline
\hline
\end{tabular}
\caption{Number of points analyzed and $\chi^2$ values for the flavor-dependent fit.}
\label{t:fl_dep_chi2}
\end{table}
%%%%%%%%%%%%%%%% Tab. common parameters %%%%%%%%%%%%%%%%%%%%%%%%%%
\begin{table}[h!]
\small
  \centering
  \begin{tabular}{|c|c|c|}
\hline
\hline
%  \multicolumn{4}{|c|}{Parameters for TMD PDFs} \\
%  \hline
%  \hline
$b_{\rm max}$ & $b_{\rm min}$ &  $g_2$ 
 \\ 
 (fixed)     & (fixed)   & {[GeV$^2$]}                           \\
\hline
$2 e^{-\gamma_E}/$GeV& $2 e^{-\gamma_E}/Q$  & $0.13 \pm 0.01$  \\
\hline
\hline
\end{tabular}
\caption{Values of parameters common to TMD PDFs and FFs (for the flavor-dependent fit).}
\label{t:fl_dep_parcommon}
\end{table}
%%%%%%%%%%%%%%% Tab. results paramters TMDs %%%%%%%%%%%%%%%%%%%%%%
\begin{table}[h!]
\small
  \centering
  \begin{tabular}{|c||c|c|c|c|c|c|}
\hline
\hline
%  \multicolumn{4}{|c|}{Parameters for TMD PDFs} \\
%  \hline
%  \hline
TMD PDFs&  $\big \langle \hat{\bm{k}}_{\T}^2 \big \rangle$ 
& $\alpha$ & $\sigma$ & & $\lambda$ &  
 \\ 
        & {[GeV$^2$]}                               &
      (random) &      &  & & \\
\hline
up valence 
& $0.15 \pm  $ & $0.00 \pm   $ & $-0.93 \pm  $  &  & $50.0 \pm  $ &
\\
\hline
down valence 
& $0.31 \pm  $ & '' & ''  &  & '' &    \\
\hline
sea 
& $0.17 \pm  $ & $4.56 \pm   $ & $0.27 \pm  $  &  & $0.147 \pm  $ &    \\
\hline
\hline
TMD FFs&  $\big \langle \hat{\bm{P}}_{\perp}^2 \big \rangle$ &
$\beta$ & $\delta$ & $\gamma$ & $\lambda_F$ & $\big \langle
\hat{\bm{P}}_{\perp}^{\prime 2} \big \rangle$
 \\ 
        & {[GeV$^2$]} &            &         & & &{[GeV$^2$]}    \\
\hline
$u \to \pi^+$   &  $0.22 \pm $ & $2.6 \pm  $ & $2.8 \pm $ 
      & $0.062 \pm $ & $5.9 \pm $ & $0.139 \pm $  \\
\hline
$d \to \pi^+$  &  $0.24 \pm $ & '' & '' & '' & '' & ''  \\
\hline
$\bar{s} \to K^+$ &  $0.24 \pm$ (random) & '' & '' & '' & '' & ''   \\
\hline
$u \to K^+$   &  $0.22 \pm $ & '' & '' & '' & '' & ''  \\
\hline
\hline
\end{tabular}
\caption{$68\%$ confidence intervals of best-fit parameters for TMD PDFs and TMD FFs in the flavor-dependent fit.}
\label{t:fl_dep_par_TMD}
\end{table}
%%%%%%%%%%%%%%%%%%%%%%%%%%%%%%%%%%%%%%%%%%%%%%%%%%%%%%%%%%%







Description of Hermes data (version from Dropbox, Feb. 9$^{th}$).
Legend for $z$ values needs to be added too.
%%%%%%%%%%%%%%%%%%%%%%%%%%%%%%%%%
\begin{figure}[h!]
\begin{center}
\includegraphics[width=0.85\textwidth]{plots/Hermes/Hermes_Pions_SCIplot_flINDEP.pdf}
\end{center}
\caption{Hermes multiplicities for production of pions off a proton and a deuteron for different $\langle x \rangle$, $\langle z \rangle$, and $\langle Q^2 \rangle$ bins as a funciton of the transverse momentum of the dected hadron ${\bm P}_{hT}^ 2$.} 
\label{f:H_pions}
\end{figure}
%%%%%%%%%%%%%%%%%%%%%%%%%%%%%%%%%
\begin{figure}[h!]
\begin{center}
\includegraphics[width=0.85\textwidth]{plots/Hermes/Hermes_Kaons_SCIplot_flINDEP.pdf}
\end{center}
\caption{Hermes multiplicities for production of kaons off a proton and a deuteron for different $\langle x \rangle$, $\langle z \rangle$, and $\langle Q^2 \rangle$ bins as a funciton of the transverse momentum of the dected hadron ${\bm P}_{hT}^ 2$.} 
\label{f:H_kaons}
\end{figure}
%%%%%%%%%%%%%%%%%%%%%%%%%%%%%%%%%



Description of Compass data (version from Dropbox, Feb. 9$^{th}$).
%%%%%%%%%%%%%%%%%%%%%%%%%%%%%%%%%
\begin{figure}[h!]
\begin{center}
\includegraphics[width=0.85\textwidth]{plots/Compass/COMPASS_SCIplot_flINDEP_Piminus.pdf}
\end{center}
\caption{Compass multiplicities for production of negative hadrons (pions) off a deuteron for different $\langle x \rangle$, $\langle z \rangle$, and $\langle Q^2 \rangle$ bins as a funciton of the transverse momentum of the dected hadron ${\bm P}_{hT}^ 2$. Multiplicities are normalized to the first bin in ${\bm P}_{hT}^ 2$ for each $\langle z \rangle$ value.} 
\label{f:C_pim}
\end{figure}
%%%%%%%%%%%%%%%%%%%%%%%%%%%%%%%%%
\begin{figure}[h!]
\begin{center}
\includegraphics[width=0.85\textwidth]{plots/Compass/COMPASS_SCIplot_flINDEP_Piplus.pdf}
\end{center}
\caption{Compass multiplicities for production of positive hadrons (pions) off a deuteron for different $\langle x \rangle$, $\langle z \rangle$, and $\langle Q^2 \rangle$ bins as a funciton of the transverse momentum of the dected hadron ${\bm P}_{hT}^ 2$. Multiplicities are normalized to the first bin in ${\bm P}_{hT}^ 2$ for each $\langle z \rangle$ value.} 
\label{f:C_pip}
\end{figure}
%%%%%%%%%%%%%%%%%%%%%%%%%%%%%%%%%





Description of low energy Drell-Yan data (version from Dropbox, Feb. 9$^{th}$).
Legends need to be added too. Fix the y-axis label.
%%%%%%%%%%%%%%%%%%%%%%%%%%%%%%%%%%
\begin{figure}[h!]
\centering
\begin{tabular}{ccc}
\includegraphics[width=0.40\textwidth]{plots/DY-Z/DY_SCIplot_flINDEP_1.pdf}
&\hspace{0.001cm}
&
\includegraphics[width=0.40\textwidth]{plots/DY-Z/DY_SCIplot_flINDEP_2.pdf}
\\
(a) && (b)
\end{tabular}
\caption{write the caption here (a) and another here (b).}
\label{f:DY_panel_1}
\end{figure}
%%%%%%%%%%%%%%%%%%%%%%%%%%%%%%%%%
\begin{figure}[h!]
\centering
\begin{tabular}{ccc}
\includegraphics[width=0.40\textwidth]{plots/DY-Z/DY_SCIplot_flINDEP_3.pdf}
&\hspace{0.001cm}
&
\includegraphics[width=0.40\textwidth]{plots/DY-Z/DY_SCIplot_flINDEP_4.pdf}
\\
(a) && (b)
\end{tabular}
\caption{write the caption here (a) and another here (b).}
\label{f:DY_panel_2}
\end{figure}
%%%%%%%%%%%%%%%%%%%%%%%%%%%%%%%%%





Description of Z-boson production data (version from Dropbox, Feb. 9$^{th}$).
Legends need to be added too. Fix the y-axis label.
%%%%%%%%%%%%%%%%%%%%%%%%%%%%%%%%%
\begin{figure}[h!]
\begin{center}
\includegraphics[width=0.85\textwidth]{plots/DY-Z/Z_SCIplot_flINDEP.pdf}
\end{center}
\caption{Cross section differential with respect to the transverse momentum $q_T$ of a $Z$ boson produced from $p\bar{p}$ collisions at Tevatron. The four panels refer to different experiments (CDF and D$0$) with two different values for the center-of-mass energy ($\sqrt{s} = 1.96$ TeV and $\sqrt{s}=1.8$ TeV).} 
\label{f:Z_qT}
\end{figure}
%%%%%%%%%%%%%%%%%%%%%%%%%%%%%%%%%











%%%%%%%%%%%%%%%%%%%%%%%%%%%%%%%%%%%%%%%%%%%%%%%%%%%%%%%%%%%%%%%%%%
\section{Conclusions and outlook}
\label{s:end}
%%%%%%%%%%%%%%%%%%%%%%%%%%%%%%%%%%%%%%%%%%%%%%%%%%%%%%%%%%%%%%%%%%



\newpage
%%%%%%%%%%%%%%%%%%%%%%%%%%%%%%%%%%%%%%%%%%%%%%%%%%%%%%%%%%%%%%%%%%
\begin{acknowledgments}
%Discussions with  are gratefully acknowledged. 
This work is supported by the European Research Council (ERC) under the European Union's Horizon 2020 research and innovation program (grant agreement No. 647981, 3DSPIN). 
AS acknowledges support from U.S. Department of Energy contract DE-AC05-06OR23177, under which Jefferson Science Associates, LLC, manages and operates Jefferson Lab. 
The work of AS has been funded partly also by the program of the Stichting voor Fundamenteel Onderzoek der Materie (FOM), which is financially supported by the Nederlandse Organisatie voor Wetenschappelijk Onderzoek (NWO).
\end{acknowledgments}
%%%%%%%%%%%%%%%%%%%%%%%%%%%%%%%%%%%%%%%%%%%%%%%%%%%%%%%%%%%%%%%%%%
%\bibliographystyle{myrevtex}
%\bibliographystyle{h-physrev}
\bibliographystyle{apsrevM}
%\bibliographystyle{JHEP}
\bibliography{mybiblio}
%\bibliography{biblio_sidis}
%\bibliography{bibroad}
%%%%%%%%%%%%%%%%%%%%%%%%%%%%%%%%%%%%%%%%%%%%%%%%%%%%%%%%%%%%%%%%%%


\end{document}

%%% Local Variables: 
%%% mode: latex
%%% TeX-master: t
%%% End: 


%% figure with one panel :
%%%%%%%%%%%%%%%%%%%%%%%%%%%%%%%%%
%\begin{figure}[h!]
%\begin{center}
%\includegraphics[width=***cm]{fig_name}
%\end{center}
%\caption{} 
%\label{f:xxxxx}
%\end{figure}
%%%%%%%%%%%%%%%%%%%%%%%%%%%%%%%%%

%%% figure with two panels :
%%%%%%%%%%%%%%%%%%%%%%%%%%%%%%%%%%
%\begin{figure}
%\centering
%\begin{tabular}{ccc}
%\includegraphics[width=***cm]{fig_name}
%&\hspace{0.001cm}
%&
%\includegraphics[width=6cm]{fig_name}
%\\
%(a) && (b)
%\end{tabular}
%\caption{write the caption here (a) and another here (b).}
%\label{f:xxxxx}
%\end{figure}
%%%%%%%%%%%%%%%%%%%%%%%%%%%%%%%%%




